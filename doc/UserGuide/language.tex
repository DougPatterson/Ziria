\chapter{\LANG language}
\label{chap:lang}

\section{Overview}
\label{sec:lang:overview}

\LANG is designed as a 2-layer language. The lower level is called an \emph{expression} language. It is an imperative language, similar to C and Matlab, for manipulating basic data types (bits, bytes, arrays, etc), described in \ref{sec:lang:expression} The higher level is
called a \emph{computational} language. It is a monadic\footnote{A monad is a structure that represents computations defined as sequences of steps} language for specifying and staging stream processor. It is described in \ref{sec:lang:computation}.

We start by giving a simple example of the scrambler in WiFi 802.11 a/g transmitter. The goal of the scrambler is to xor an input sequence with a predefined pseudo-random sequence (to make the transmitter input look random). \LANG code for the scrambler is given below:
\begin{lstlisting}[caption=Scrambler function of WiFi 802.11a/g transmitter in \LANG,
xleftmargin=15pt,
label=lst:wifi:tx:scrambler]
let scrambler() =
  @\label{scrambler:letref}@var scrmbl_st: arr[7] bit := {1,1,1,1,1,1,1}; 
  @\label{scrambler:letref2}@var tmp: bit; 
  @\label{scrambler:letref3}@var y: bit; 
@\label{scrambler:repeat}@  in repeat (
@\label{scrambler:take}@    x <- take;
@\label{scrambler:ret_start}@    execute (
      tmp := (scrmbl_st[3] ^ scrmbl_st[0]);
@\label{scrambler:shift}@      scrmbl_st[0:5] := scrmbl_st[1:6];
@\label{scrambler:feed}@      scrmbl_st[6] := tmp;
@\label{scrambler:ret_end}@      y := x ^ tmp); 
@\label{scrambler:emit}@    emit (y))
\end{lstlisting}
The scrambler is written as a sequence of operations in the
computational language. The scrambler's body declares three local
variables in lines~\ref{scrambler:letref} through
\ref{scrambler:letref2}: \lstinline|scrmbl_st|, an array of $7$ bits
that gives the current state of the shift register, and two one-bit
references: \lstinline|tmp| and \lstinline|y|.  The scrambler
\lstinline|take|s a value from the input stream
(line~\ref{scrambler:take}), then performs an imperative computation
that assigns \lstinline|tmp| the XOR of taps $3$ and $0$ in the shift
register, shifts the register state left by one, feeds \lstinline|tmp|
into position $6$ of the register, and finally returns the XOR of
\lstinline|tmp| and the input bit \lstinline|x|.

The sequence of operations inside \lstinline|execute()| primitive is written in the \LANG expression language. \LANG\ allows for arbitrary composition of the stream 
computer primitives and imperative code to create complex programs.
In the subsequent section we will describe in more detail the expression and the computation languages.



\section{Expression language}
\label{sec:lang:expression}

The low level imperative language for \LANG is used to express the basic imperative
calculation. This language is a mix of Matlab and C designed to make
the right trade-off between programmability and efficiency. It is a
strongly typed language, which allows us to simplify memory
management. It also supports array operations, like Matlab, which
allows for efficient mapping to SSE vector instructions.  






\section{Computational language}
\label{sec:lang:computation}

asdada
